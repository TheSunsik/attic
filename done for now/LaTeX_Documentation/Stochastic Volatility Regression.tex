\documentclass[10pt]{beamer}

%% basic metropolis theme setting
\usetheme[progressbar=frametitle]{metropolis}
%In using metropolis theme, we can add progressbar like this.
\setbeamertemplate{frame numbering}[fraction]
\useoutertheme{metropolis}
\useinnertheme{metropolis}
\usefonttheme{metropolis}
\usecolortheme{spruce}
\setbeamercolor{background canvas}{bg=white}
%%%%%

\title{Bayesian Functional Data Modelling for Heterogeneous Volatility}
\subtitle{On Stochastic Volatility Regression}
\author{\large{\textbf{Sunsik Kim}}}
%\institute{we can use this space to \\[6pt]\textbf{display other information}}
\date{}

%%% Define frequently used expressions
\def\mean{M_{k_i}(t)}
\def\dev{U_i(t)}
\def\vol{\sigma_{U_i}^2}
%%%%%

\usepackage{amsfonts}
%\usepackage{mathrsfs}
%Load this package if you need really cursive-formal letter
%Use with \mathscr command

\begin{document}
\metroset{block=fill}
\maketitle


\begin{frame}[t]{Why SVR}
\vspace{4pt}
\begin{itemize}
\item In FDA literature, variability among volatilities between individual functions has been somewhat neglected, even though such variability might contain important information.
\item To include the effect of individual variability, novel class of functional data analysis models characterized using hierarchical stochastic differential equations is presented in this paper.
\end{itemize}
\end{frame}


\section{Model Specification}
\begin{frame}[t]{Observation equation}
\vspace{4pt}
We specify an observation equation for $Y_i(t)$ as:
\begin{center}
$Y_i(t)=\mean+\dev+\varepsilon_i(t)$
\end{center}
\begin{itemize}
\item $\varepsilon_i(t)$ : measurement error at time $t$(follows $N(0,\sigma^{2}_{\varepsilon})$)
\item $\mean$ : Value of $k_i$th group mean function at time $t$\\
(calculated as $E\{Y_i(t)|M_{k_i}(t)\}$)
% I think it should be $E\{Y_i(t)|k_i, t\}$ but let's pass for now
\item $\dev$ : Subject-specific deviation from the group mean at time $t$\\
(individual volatility $\vol$ is defined using this deviation function)
\end{itemize}
\end{frame}


\begin{frame}[t]{Volatility}
\vspace{4pt}
\begin{block}{Volatility}
\vspace{0.5em}
With defferential operator $D^q=\displaystyle\frac{d^q}{dt^q}$, we denote\\
$$\vol=\lim\limits_{h \to 0}\displaystyle\frac{1}{h}E[\{D^{q-1}U_i(t+h)-D^{q-1}\dev\}^2|D^{q-1}\dev]$$
\vspace{0.5em}
\end{block}
The volatility $\vol$ is allowed to vary through simple Gaussian log linear model(i.e. $log(\vol)\sim N_1(x_{i}^{T}\beta,\sigma^2)$). This implies that \textbf{volatility}
\begin{enumerate}
\item is \textbf{constant over time}(doesn't rely on $t$)
\item \textbf{varies across subjects}(indexed by $i$, which refer to individual observation)
\item \textbf{depends on covariates}(relies on $x_i$)
\end{enumerate}
\end{frame}


\begin{frame}[t]{Priors for mean, deviance function}
\vspace{4pt}
Gaussian process priors for mean and deviance function is specified by using \textbf{stochastic differential equations} as following:
\begin{center}
    $D^pM_{k_i}(t)=\sigma_{M_{K_i}}\dot{W}_{k_i}(t)$\\
    $D^qU_i(t)=\sigma_{U_i}\dot{W}_{i}^{'}(t)$
\end{center}
Note that $p, q \in \mathbb{N}$\ and\  $\sigma_{M_{k_i}}, \sigma_{U_i} \in \mathbb{R}^+$ where $\sigma_{M_{k_i}}, \sigma_{U_i}$ refer to group and individual volatility respectively. Also, $\dot{W}_{k_i}(t)$ and $\dot{W}_{i}^{'}(t)$ are independent Gaussian white noise process with
\begin{center}
    $E\{\dot{W}_{k_i}(t)\}=E\{\dot{W}_{i}^{'}(t)\}=0$\\
    $E\{\dot{W}_{k_i}(t)\dot{W}_{k_i}(t^{'})\}=E\{\dot{W}_{i}^{'}(t)\dot{W}_{i}^{'}(t^{'})\}=\delta(t-t^{'})$
\end{center}
\end{frame}


\begin{frame}[t]{Priors for mean, deviance function}
\vspace{4pt}
Accordingly, mean and covariance functions of $\mean$ and $\dev$ is obtained by applying \textbf{stochastic integration} to stochastic differential equations. Such process result in hierarchical Gaussian process of
\begin{align*}
\mean +\dev | & \mean \sim GP(\mean , K_{U_{i0}}(s,t)+K_{U_{i1}}(s,t)),\\
	& \mean \sim GP(0 , K_{M_{k_{i}0}}(s,t)+K_{M_{k_{i}1}}(s,t))
\end{align*}
where each covariance function is composed as
\begin{align*}
K_{U_{i0}}(s,t)+K_{U_{i1}}(s,t)& =\sigma^{2}_{U_0}\mathcal{R}_{U_0}(s,t)+\sigma^{2}_{U_i}\mathcal{R}_{U_1}(s,t)\\
K_{M_{k_{i}0}}(s,t)+K_{M_{k_{i}1}}(s,t)& =\sigma^{2}_{M_0}\mathcal{R}_{M_0}(s,t)+\sigma^{2}_{M_{k_i}}\mathcal{R}_{M_1}(s,t).
\end{align*}
Note that covariance function of obtained Gaussian Process depends on covariates of specific individuals through $\sigma^{2}_{U_i}$ whose log value follows $N(x_{i}^{T}\beta, \sigma_{\varepsilon})$ distribution.
\end{frame}


\begin{frame}[t]{Priors for mean, deviance function}
\vspace{4pt}
Specific form of components in covariance functions presented in previous slide is presented as below:
\begin{align*}
\mathcal{R}_{U_0}(s,t)=\sum_{l=0}^{q-1}\phi_{l}(s)\phi_{l}(t),\ \mathcal{R}_{U_1}(s,t)=\int_{\tau}G_q(s,u)G_q(t,u)\ du\\
\mathcal{R}_{M_0}(s,t)=\sum_{l=0}^{p-1}\phi_{l}(s)\phi_{l}(t),\ \mathcal{R}_{M_1}(s,t)=\int_{\tau}G_p(s,u)G_p(t,u)\ du
\end{align*}
where $\displaystyle\phi_l(t)=\frac{t^l}{l!},\ G_k(x,u)=\frac{(x-u)^{k-1}}{(k-1)!}\ and\ s,t,u \in \tau$
\end{frame}


\begin{frame}[t]{Posterior mean if mean, deviance function}
\vspace{4pt}

\end{frame}


\begin{frame}[standout]
\flushleft
Thank you
\end{frame}

\end{document}