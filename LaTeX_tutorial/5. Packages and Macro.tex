\documentclass[11pt]{article}

\usepackage{fullpage}
\usepackage{amsfonts}
\usepackage{graphicx}
\def\eq1{y=\frac{x}{3x^2+x+1}}

\begin{document}
We can add extra functionality to Texmaker using some packages. They include:

\begin{itemize}
fullpage, geometry : setting page layout(controlling page margin)
amsfonts : change letter into special form(ex. set of natural numbers)
graphicx : insert image into the document
\end{itemize}

We write command that includes package in preamble(space between documentclass command
and begin document command) using "usepackage" command. Insert the name of the package
in curly bracket, and insert some options in square bracket, if necessary. For example,
code above load fullpage package. This package set margin automatically and gives tidy 
paper layout. 

If we want to manually specify the size of the margin, we load "geometry" package. We 
can specify certain options like following:

usepackage[top=1cm, bottom=1cm, left=0.5cm, right=0.5cm]{geometry}\\
usepackage[margin=1cm, paperwidth=8cm, paperheight=11cm]{geometry}

"amsfonts" package is used to insert symbols that are mainly used to specify certain set.
For example, we can include sign for the set of natural numbers like $\mathbb{N}$, given
that we've loaded amsfonts package.

"graphicsx" package gives functionality that enable user to include image that is in
the same directory with current .tex file. Note that only png, jpg, gif, pdf file can
be included, and space in file name is forbidden. We include image as such:
\begin{center}
\includegraphics[width=5in, height=8in]{filename.extension}
\end{center}

We can also include graphic in its original weight-height proportion. We specify scale
value to scale down/up a input image.
\includegraphics[scale=0.5]{filename.extension}

We can also rotate image if we specify angle:
\includegraphics[angle=45]{filename.extension}

Macro is used to build custom LaTeX command. It is useful when we need to type in same
expression many times in a document. By "def" command and macro's name, we can define 
that name to designate certain text. For example, we define a macro named as 'eq1' above.
Then we can use this macro like \eq1 or $\eq1$(if we want to include this in math mode).

\end{document}