\documentclass[11pt]{article}

\begin{document}

To make an ordered list with numbers attached:
\begin{enumerate}
\item pencil
\item calculator
\item ruler
\item notebook
\item graph paper
\end{enumerate}

To make an nested list with numbers attached, run following code. Note that added
tab key in front of the code for sublist is optional; it's just added to make code
look tidier(to distinguish parent, child list apparently).
\begin{enumerate}
\item notebook
	\begin{enumerate}
	\item Math
		\begin{enumerate}
		\item Analysis
		\item Matrix Algebra
		\end{enumerate}
	\item Literature
	\item Science
	\end{enumerate}
\item graph paper
\end{enumerate}
Mixture of two types of lists is also possible.

To make an unordered list, execute:
\begin{itemize}
\item notebook
	\begin{enumerate}
	\item Math
	\item Literature
	\item Science
	\end{enumerate}
\item graph paper
\item calculator
\end{itemize}

To specify a symbol used in lists, use square brackets like:
\begin{enumerate}
\item[Commutative] $a+b=b+a$
\item[Associative] $a+(b+c)=(a+b)+c$
\item[Distributive] $a(b+c)=ab+ac$
\end{enumerate}

\end{document}