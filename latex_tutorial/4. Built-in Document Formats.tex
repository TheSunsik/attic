\documentclass[11pt]{article}

\begin{document}
\title{This is title section}
\author{Sunsik Kim}
\date{\today}
\maketitle
Above command creates specified information about the document in the format of title.
We can exclude some of them, but can't exclude maketitle command; if so, title part isn't generated.

\tableofcontents
Since we specified sections and subsections below, this simple command will build table
of contents accordingly. Note that we need to execute quick build twice to properly
return the desired table of contents.

This will produce \textit{italicized} text. 
This will produce \textbf{bold-faced} text.
This will produce \textsc{small caps} text. That is, this returns upper case
letters in the size of lower case letters.
This will produce \texttt{typewriter} font. This a font used in R markdown code
chunk.

Increase fontsize 
\begin{large} like this \end{large}. 
For larger fontsize, use 
\begin{LARGE} this \end{LARGE} or 
\begin{huge} this \end{huge} or 
\begin{Huge} this \end{Huge}.

In case we want to decrease the fontsize, use 
\begin{small} this \end{small} or 
\begin{tiny} this \end{tiny}.

\begin{center}
To center certain part of the text, use this.
\end{center}

\begin{flushleft}
To align certain part of the text to the left, use this.
\end{flushleft}

\begin{flushright}
To align certain part of the text to the right, use this.
\end{flushright}

\section{Linear Functions}
This is reserved layout for specifying certain chapter.
	\subsection{Slope-Intercept From}
	This is also a reserved layout.
	\subsection{Standard Form}
	Numbering will be applied automatically by the compiler.
	\subsection{Point-Slope Form}
\section{Quadratic Functions}
	\subsection{Vertex Form}
	\subsection{Standard Form}
	\subsection{Factored Form}

\end{document}