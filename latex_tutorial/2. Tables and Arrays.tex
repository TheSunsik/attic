\documentclass[11pt]{article}

\begin{document}
Curly brackets are reserved symbol, so we need backslash to display this symbol to
the pdf. This is true for any other reserved symbols.
$$\{curly\ brackets\}$$
$$\$dollar\ sign\$$$\\

To extend wrapper to match a height of an object, we need "left" or "right". For
example, if we want to write Jacobian of univariable pdf transformation, we would
want to extend vertical bar in the expression:
$$|\frac{dx}{dy}|$$

in order to express this symbol in the form of:
$$\left| \frac{dx}{dy} \right|$$

However, sometimes we might not need the left bar when we want to express derivative of $y=f(x)$ evaluated at $x=x_{0}$. Then make use of reserved symbol period:
$$\left. \frac{dy}{dx} \right|_{x=x_{0}}$$

Now we will build table. At the second curly bracket, we type the form of
alignment
in desired number of times. If we want six columns to be aligned center, we type:
\begin{tabular}{cccccc}
$x$ & 1 & 2 & 3 & 4 & 5 \\
$f(x)$ & 3 & 5 & 7 & 9 & 11
\end{tabular}

Note that we move to next column entry by writing ampersand and move to next row
by writing double backslash. Moreover, if we want to add a horizontal line between
rows or add vertical line between columns, we perform:
\begin{tabular}{|c|ccccc|}
\hline
$x$ & 1 & 2 & 3 & 4 & 5 \\ \hline
$f(x)$ & 3 & 5 & 7 & 9 & 11 \\ \hline
\end{tabular}

Now we make array by begin equationarray(eqnarray). Note that (1)math mode is
automatically set in eqnarray chunk; we don't need to specify dollar sign. Also
(2)we change line by writing double backslashes, and designate aligner by wrapping
the symbol to use as aligner with ampersand. Finally, (3)to hide line numbers that
is attached as default, we attach asterisk like below:
\begin{eqnarray*}
5x^2-9&=&x+3\\
4x^2&=&12\\
x^2&=&3\\
x&\approx&\pm1.732
\end{eqnarray*}


\end{document}